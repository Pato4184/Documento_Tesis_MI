\chapter{Introducción}
\label{intro}
%\chapterquote{Hablaban siempre de dinero y planeaban asaltar un banco}{Domingo Cavallo, 2001}

\section{\textcolor{red}{Tesis Maestría Machaca}}

En la actualidad muchos problemas de la ingeniería presentan flujos en régimen de transición. Ejemplo de esto son las alas de los aviones, los  álabes de las turbinas, los intercambiadores de calor, entre otros. 

Desde el punto de vista ingenieril, aunque este es un régimen de trabajo no deseado por su carácter intermitente, en el cual el flujo puede fluctuar entre los regímenes laminar y turbulento, las características de este fenómeno resultan de gran relevancia, ya que el coeficiente de fricción \cite{white} y el coeficiente de convección \cite{incropera} se incrementan notablemente al pasar del régimen laminar al turbulento \cite{tam2006transitional}.

Por otro lado, un flujo se encuentra en un estado de transición desarrollado cuando no varía con el tiempo ni con el espacio en un promedio estadístico; en este caso, se dice que el flujo está en régimen de transición. La evolución de un flujo laminar a un flujo turbulento completamente desarrollado se denomina transición laminar-turbulenta, y puede ocurrir en el tiempo, en cuyo caso se habla de transición laminar-turbulenta temporal, o en el espacio, lo que se conoce como transición laminar-turbulenta espacial.


En las últimas décadas, se han realizado numerosos esfuerzos para desarrollar técnicas que mejoren la transferencia de calor y el desempeño global de los intercambiadores de calor, motivados principalmente por el interés en ahorrar energía. Con este objetivo, se han llevado a cabo experimentos tanto en tubos como en canales, con el fin de determinar experimentalmente correlaciones de transferencia de calor \cite{hausen1959new, gnielinski1976new, churchill1977comprehensive, sleicher1975convenient}.

Por otro lado, el estudio de la transferencia de calor en canales rectangulares ha ganado interés en los últimos años, motivado por su aplicación en reactores nucleares \cite{sikorska2024convective}, en el área de sistemas electrónicos avanzados por el sistema de refrigeración \cite{kamdem2020numerical}

Es importante destacar que los experimentos reales suelen ser costosos, por lo que a menudo se recurre a experimentos numéricos como alternativa o complemento. En el campo de la simulación numérica de fluidos, surge el concepto de fluidodinámica computacional, que mediante simulaciones numéricas busca explicar el comportamiento de los fluidos. Sin embargo, es sabido que el régimen turbulento presenta un comportamiento caótico y fluctúa rápidamente en el espacio y el tiempo, lo que hace su estudio numérico complejo. Aún más desafiante es el análisis del flujo en transición, que representa un estado intermedio entre el régimen laminar y el turbulento.

Hoy en día, el uso de supercomputadoras para resolver las ecuaciones que describen el movimiento de un fluido ha ganado relevancia y se ha convertido en una herramienta clave para el estudio de flujos turbulentos y en transición. Gracias al avance de las computadoras de alto rendimiento, la simulación numérica directa (DNS) se ha convertido en una herramienta esencial para investigar la turbulencia y la transición. El DNS permite calcular la solución tridimensional y dependiente del tiempo de las ecuaciones de conservación de masa, momento y energía. Como estas ecuaciones se resuelven sin un modelo de turbulencia, requieren una malla computacional fina para capturar todas las escalas del flujo. A medida que el número de Reynolds aumenta, surgen escalas más pequeñas \cite{pope2001turbulent}, lo que demanda mallas aún más finas para una correcta representación.

Además, la simulación numérica directa del transporte de un escalar pasivo, como la temperatura, en un flujo turbulento requiere especial atención, ya que a un número de Reynolds fijo, el aumento en el número de Prandtl incrementa el requerimiento de mallado para capturar adecuadamente las variaciones de temperatura.

Una de las primeras simulaciones numéricas directas de flujo turbulento completamente desarrollado fue realizada por Kim y Moin \cite{kim1989transport}. Más adelante, con el apoyo de la computación en paralelo masiva, Kawamura et al. \cite{kawamura2000dns} levaron a cabo simulaciones DNS en un canal periódico. %El estudio numérico de la transición es aún más complejo, ya que es necesario inestabilizar el flujo para desencadenar el proceso de transición. En particular, en el caso de la transición laminar-turbulenta espacial, se requiere el uso de dominios muy largos en la dirección de la corriente para capturar el proceso adecuadamente \cite{Schlatter2005, Schmid2001}.

%Un aspecto importante a considerar son las características numéricas de la herramienta utilizada en este estudio. El método DNS, en general, requiere esquemas numéricos precisos para representar adecuadamente el amplio rango de escalas espaciales y temporales de los problemas de interés, como la transición y la turbulencia, con un costo computacional adecuado. Si bien los métodos espectrales son deseables debido a su alta precisión de convergencia, su aplicación se limita en general a dominios simples o académicos y a condiciones de contorno específicas, como canales o cajas periódicas, y a grillas uniformes. En respuesta a estas limitaciones, los esquemas compactos de diferencias finitas de alto orden tipo Padé \cite{Lele1992} representan una alternativa válida para capturar las escalas espaciales con un costo computacional razonable. Estos esquemas permiten flexibilidad en el tratamiento de las condiciones de contorno en grillas tanto uniformes como no uniformes y son relativamente simples en su planteamiento.

%Este tipo de código se ha consolidado como una poderosa herramienta numérica para la investigación académica \cite{Laizet2010, Lamballais2011}.


\section{\textcolor{red}{Turbulent Flows - Pope}}

\subsection{Cap 1}

La principal motivación para el estudio de los flujos turbulentos radica en la combinación de tres factores clave: en primer lugar, la mayoría de los flujos en la naturaleza son turbulentos; en segundo lugar, el transporte y la mezcla de materia, momento y calor en estos flujos son de gran importancia práctica; y en tercer lugar, la turbulencia incrementa significativamente las tasas de estos procesos. 

El primer paso para clasificar estos estudios es diferenciar entre la turbulencia a pequeña escala y los movimientos a gran escala en los flujos turbulentos. Mientras que los movimientos a gran escala están fuertemente influenciados por la geometría del flujo, es decir, por las condiciones de contorno, el comportamiento de las turbulencias a pequeña escala está determinado casi exclusivamente por la cantidad de energía que reciben de los movimientos a gran escala y por la viscosidad. Esto hace que las turbulencias a pequeña escala tengan un carácter universal, independiente de la geometría del flujo.

\textcolor{red}{Esto es una modificación}
