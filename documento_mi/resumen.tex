\begin{resumen}%

%La convección mixta en canales y conductos verticales se encuentra presente en muchos sistemas de interés, entre ellos los intercambiadores de calor. Este tipo de sistemas pueden presentar cambios de régimen (laminar-turbulento) en su funcionamiento. Desde el punto de vista ingenieril, debido a que parámetros del flujo como el coeficiente de fricción o el número de Nusselt presentan una gran variación. Por ello, en el presente trabajo se estudia la evolución temporal de magnitudes relevantes durante la transición del flujo laminar hacia el turbulento. En particular, se analiza la transición temporal laminar-turbulenta para un canal vertical de placas paralelas sometidos a un flujo de calor constante en las paredes. 
%
%Para estudiar la transición temporal primero resulta necesario conocer el estado inicial laminar
%y el estado final turbulento. En consecuencia, se analizó el flujo turbulento completamente
%desarrollado bajo la influencia de la fuerza boyante. Tanto el estudio del caso completamente desarrollado como la transición temporal se relizó vía simulación numérica directa (DNS) empleando la herramienta numérica Xcompact3D. A partir de estos resultados se estudian magnitudes clave como la energía cinética turbulenta (TKE), la varianza de la temperatura, el número de Nusselt (Nu) y el factor de fricción de Darcy ($f$).  
%
%Para inducir la inestabilidad del flujo laminar hacia uno turbulento se construyen condiciones iniciales empleando análisis de estabilidad lineal. Esto implica resolver un problema de autovalores y autofunciones derivado de las ecuaciones de Orr-Sommerfeld para convección mixta. La resolución de las misma se hace con la herramienta OSCM desarrollada previamente en el grupo (\href{https://github.com/Pato4184/OSMC-Repository}{OSMC-Repository}). Sus soluciones (o combinación de ellas) permiten desencadenar la transición de flujo y comparar la evolución temporal de magnitudes de interés como las mencionada anteriormente. 
%
%En el análisis del régimen turbulento completamente desarrollado, se consideró los número de Reynolds (basado en la velocidad en centro del canal) tales que $ 2100 \leq \text{Re}_o \leq 5000$, los números de Prandtl $\text{Pr}=0\text{.}071,0\text{.}71$ y los números de Richardson tales que $0\text{.}04 \leq \text{Ri}_b \leq 106\text{.}5$. Se emplean aquellos casos con  $\text{Re}_o =5000$, $\text{Pr}=0\text{.}71$ para analizar magnitudes estadísticas de primer orden (perfiles de velocidad y temperatura) y magnitudes de segundo orden (fluctuaciones de velocidad y temperatura, y flujo turbulento de calor). Las estimaciones del número de Nusselt obtenido a partir de nuestras simulaciones concuerdan con correlaciones de la literatura en los rangos de Re, Pr y Ri$_b$ considerados. En particular se encuentró una región ($10^{-6} \lesssim \text{Bo} \lesssim 3 \times 10^{-5}$ siendo Bo el número de boyancia) en la transferencia de calor se reduce respecto al caso puro de convección forzada. Esto último se relaciona a una disminución en la producción de turbulencia cerca las paredes. En el caso del factor de Darcy se propone una correlación propia que muestra buen acuerdo tanto para datos propios como de referencias de otros autores.
%
%Se abordó la transición laminar-turbulenta con $\text{Re}_o=5000$ y $\text{Pr}=0\text{.}71$, considerando dos intensidades de boyancia caracterizadas por $\text{Ri}_b=0\text{.}04$ y $\text{Ri}_b=1\text{.}06$. Se diseñaron condiciones iniciales basadas en el mecanismo de inestabilización propuesto y se identificaron combinaciones de perturbaciones que efectivamente inducen la transición, mostrando que el aumento de la boyancia incrementa la inestabilidad y adelanta la transición para ciertos tipos de perturbaciones. Durante la evolución temporal, los perfiles de velocidad y temperatura presentan pérdidas momentáneas de simetría asociadas a distribuciones no homogéneas de vórtices cerca de las paredes. Las magnitudes TKE, varianza de la temperatura, Nu y $f$ exhiben estados transitorios no monótonos; en el caso con mayor $\text{Ri}_b$ se observa una caída brusca de Nu coincidente con un aumento de TKE y con el aplanamiento del perfil de velocidad producto de la difusión de momento por efecto de la turbulecnia. Aunque la boyancia acelera el desarrollo térmico, en el tiempo simulado Nu no alcanzó el régimen completamente desarrollado en ninguno de los dos casos. Finalmente, la evolución temporal de $\text{Re}_{\tau}$ culmina por encima del valor inicial en el caso de menor $\text{Ri}_b$ y por debajo en el de mayor $\text{Ri}_b$, lo que se explica por cambios del gradiente de velocidad próximo a la pared vinculados al aplanamiento de los perfiles y al incremento súbito de turbulencia.

La convección mixta en canales y conductos verticales está presente en numerosos sistemas de interés, entre ellos los intercambiadores de calor. Estos sistemas pueden presentar cambios de régimen (laminar-turbulento) durante su funcionamiento. Desde el punto de vista ingenieril, esto es relevante porque parámetros del flujo como el coeficiente de fricción o el número de Nusselt pueden experimentar grandes variaciones. Por ello, en el presente trabajo se estudia la evolución temporal de magnitudes relevantes durante la transición del flujo laminar hacia el turbulento. En particular, se analiza la transición temporal laminar-turbulenta para un canal vertical de placas paralelas sometido a un flujo de calor constante en las paredes.

Para estudiar la transición temporal, primero resulta necesario conocer el estado inicial laminar y el estado final turbulento. En consecuencia, se analizó el flujo turbulento completamente desarrollado bajo la influencia de la fuerza boyante. Tanto el estudio del caso completamente desarrollado como la transición temporal se realizó vía simulación numérica directa (DNS) empleando la herramienta numérica Xcompact3D. A partir de estos resultados se estudiaron magnitudes clave como la energía cinética turbulenta (TKE), la varianza de la temperatura, el número de Nusselt (Nu) y el factor de fricción de Darcy ($f$).

Para inducir la inestabilidad del flujo laminar hacia uno turbulento, se construyeron condiciones iniciales empleando análisis de estabilidad lineal. Esto implica resolver un problema de autovalores y autofunciones derivado de las ecuaciones de Orr-Sommerfeld para convección mixta. La resolución de las mismas se realiza con la herramienta OSMC desarrollada previamente en el grupo (\href{[https://github.com/Pato4184/OSMC-Repository}{OSMC-Repository}). Sus soluciones (o combinaciones de ellas) permiten desencadenar la transición del flujo y analizar la evolución temporal de magnitudes de interés como las mencionadas anteriormente.

En el análisis del régimen turbulento completamente desarrollado, se consideraron los números de Reynolds (basados en la velocidad en el centro del canal) tales que $ 2100 \leq \text{Re}_o \leq 5000$, los números de Prandtl $\text{Pr}=0\text{.}071,0\text{.}71$ y los números de \linebreak Richardson tales que $0\text{.}04 \leq \text{Ri}_b \leq 106\text{.}5$. Se emplearon los casos con $\text{Re}_o =5000$, $\text{Pr}=0\text{.}71$ para analizar magnitudes estadísticas de primer orden (perfiles de velocidad y temperatura) y de segundo orden (fluctuaciones de velocidad y temperatura y flujo turbulento de calor). Las estimaciones del número de Nusselt obtenidas a partir de nuestras \linebreak simulaciones concuerdan con correlaciones de la literatura en los rangos de Re, Pr y $\text{Ri}_b$ considerados. En particular, se encontró una región ($10^{-6} \lesssim \text{Bo} \lesssim 3 \times 10^{-5}$, siendo Bo el número de boyancia) en la que la transferencia de calor se reduce respecto del caso puro de convección forzada. Esto último se relaciona con una disminución en la producción de turbulencia cerca de las paredes. Además, se propone una correlación para el factor de Darcy que muestra buen acuerdo tanto con nuestros datos como con referencias de otros autores.

Se abordó la transición laminar-turbulenta con $\text{Re}_o=5000$ y $\text{Pr}=0\text{.}71$, considerando dos intensidades de la fuerza boyante caracterizadas por $\text{Ri}_b=0\text{.}04$ y $\text{Ri}_b=1\text{.}06$. Se diseñaron condiciones iniciales basadas en el mecanismo de inestabilización propuesto y se identificaron combinaciones de perturbaciones que efectivamente inducen la transición, mostrando que el aumento de la flotación incrementa la inestabilidad y adelanta la transición para ciertos tipos de perturbaciones. Durante la evolución temporal, los perfiles de velocidad y temperatura presentan pérdidas momentáneas de simetría asociadas a distribuciones no homogéneas de vórtices cerca de las paredes. Las magnitudes TKE, varianza de la temperatura, Nu y $f$ exhiben estados transitorios no monótonos. En el caso con mayor $\text{Ri}_b$ se observa una caída brusca de Nu, coincidente con un aumento de TKE y con el aplanamiento del perfil de velocidad producto de la difusión de momento por efecto de la turbulencia. Aunque la flotación acelera el desarrollo térmico, en el tiempo simulado Nu no alcanzó el régimen completamente desarrollado en ninguno de los dos casos. Finalmente, la evolución temporal de $\text{Re}_{\tau}$ culmina por encima del valor inicial en el caso de menor $\text{Ri}_b$ y por debajo en el de mayor $\text{Ri}_b$, lo que se explica por cambios del gradiente de velocidad próximo a la pared, vinculados al aplanamiento de los perfiles y al incremento súbito de turbulencia.



\end{resumen}


\begin{abstract}%
\textcolor{red}{NOTA: Hasta que mis directores no le den el okay al resumen en español esta parte no la voy a escribir ...}
\end{abstract}



%%% Local Variables: 
%%% mode: latex
%%% TeX-master: "template"
%%% End: 
