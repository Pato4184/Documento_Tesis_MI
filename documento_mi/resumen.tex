\begin{resumen}%

La convección mixta en conductos verticales está presente en numerosos sistemas de interés, entre ellos los intercambiadores de calor. Estos sistemas pueden presentar cambios de régimen (laminar-turbulento) durante su funcionamiento. Esto es relevante ya que el coeficiente de fricción ($f$) o el número de Nusselt (Nu) pueden experimentar grandes variaciones. En el presente trabajo se estudia la evolución temporal de magnitudes de interés durante la transición desde el régimen laminar hacia el turbulento, para un canal vertical de placas paralelas con flujo de calor constante en las paredes.

Para estudiar la transición temporal, primero resulta necesario conocer el estado inicial laminar y el estado final turbulento. En ese sentido, se analiza el flujo turbulento completamente desarrollado bajo la influencia de la fuerza boyante. Tanto el estudio del estado turbulento desarrollado como la transición temporal laminar-turbulenta se realiza mediante simulaciones DNS empleando Xcompact3D. 

Las condiciones iniciales que inducen la inestabilidad se obtienen mediante análisis de estabilidad lineal, resolviendo el problema de autovalores y autofunciones derivado de las ecuaciones de Orr-Sommerfeld para convección mixta. Esta resolución se realiza con la herramienta OSMC desarrollada en el grupo MECOM. Las soluciones permiten desencadenar la transición para examinar su evolución temporal. 


En el análisis del régimen turbulento desarrollado se consideran $2100 \leq \mathrm{Re}_o \leq 5000$ \textcolor{black}{(número de Reynolds basado en el semiancho del canal y la velocidad en el centro)}, \linebreak $\mathrm{Pr}=0\text{.}071,0\text{.}71$ \textcolor{black}{(número de Prandtl)} y $0\text{.}04 \leq \mathrm{Ri}_b \leq 106\text{.}5$  \textcolor{black}{(número de \linebreak Richardson  basado en el ancho del canal y la velocidad \textit{bulk})}. Para $\mathrm{Re}_o=5000$ y $\mathrm{Pr}=0\text{.}71$ se  analizan perfiles de cantidades medias y fluctuaciones. Las estimaciones de Nu  concuerdan con correlaciones de la literatura en los rangos estudiados. \textcolor{black}{Se identifica una región del número de boyancia ($10^{-6} \lesssim \text{Bo} \lesssim 3 \times 10^{-5}$), el cual cuantifica la relación entre las fuerzas boyantes y la fuerza impulsora de la convección forzada,} donde Nu disminuye respecto de la convección puramente forzada, asociado a una menor producción de turbulencia próxima a las paredes. Se propone asimismo una correlación empírica para el factor de Darcy ($f$) con buen acuerdo frente a datos y referencias.

La transición laminar-turbulenta se examina para $\mathrm{Re}_o=5000$, $\mathrm{Pr}=0\text{.}71$ y dos intensidades de boyancia ($\mathrm{Ri}_b=0\text{.}04$ y $\mathrm{Ri}_b=1\text{.}06$). Se identifican combinaciones de perturbaciones que inducen la transición y se observa que una mayor boyancia incrementa la inestabilidad del flujo. Las cantidades TKE \textcolor{black}{(Energía Cinética Turbulenta)}, varianza de temperatura, Nu y $f$ muestran respuestas transitorias no monótonas; en el caso de mayor $\mathrm{Ri}_b$ se registra una caída pronunciada de Nu concurrente con aumento de TKE y el aplanamiento del perfil de velocidad por efecto de la turbulencia. Finalmente, se observa que el valor del estado turbulento desarrollado de $\mathrm{Re}_\tau$ \textcolor{black}{(número de Reyndols de fricción)} se encuentra por encima del valor asociado al estado inicial en el caso de $\mathrm{Ri}_b$ menor y por debajo en el de $\mathrm{Ri}_b$ mayor.

\end{resumen}


\begin{abstract}%

Mixed convection in vertical ducts is present in numerous systems of interest, including heat exchangers. These systems may undergo regime changes (laminar–turbulent) during operation. This is relevant since the friction coefficient ($f$) or the Nusselt number (Nu) may exhibit large variations. In the present work, the temporal evolution of quantities of interest is studied during the transition from the laminar regime to the turbulent one, for a vertical parallel-plate channel with constant wall heat flux.

To investigate the temporal transition, it is first necessary to characterize both the initial laminar state and the final turbulent state. In this regard, the fully developed turbulent flow under the influence of buoyancy is analyzed. Both the study of the developed turbulent state and the laminar–turbulent temporal transition are carried out through DNS simulations using Xcompact3D. 

The initial conditions that trigger the instability are obtained by means of linear stability analysis, solving the eigenvalue and eigenfunction problem derived from the Orr–Sommerfeld equations for mixed convection. This resolution is performed with the OSMC tool developed in the MECOM group. The solutions allow the transition to be triggered in order to examine its temporal evolution. 



In the analysis of the developed turbulent regime, $2100 \leq \mathrm{Re}_o \leq 5000$ \textcolor{black}{(Reynolds number based on the channel half-width and the centerline velocity)}, \linebreak $\mathrm{Pr}=0\text{.}071,0\text{.}71$ \textcolor{black}{(Prandtl number)} and $0\text{.}04 \leq \mathrm{Ri}_b \leq 106\text{.}5$ \textcolor{black}{(Richardson number based on the channel width and the bulk velocity)} are considered. For $\mathrm{Re}_o=5000$ and $\mathrm{Pr}=0\text{.}71$, mean and fluctuating quantities are analyzed. The estimates of Nu agree with correlations from the literature within the studied ranges. \textcolor{black}{A region of the buoyancy number ($10^{-6} \lesssim \text{Bo} \lesssim 3 \times 10^{-5}$), which quantifies the relationship between buoyancy forces and the driving force of forced convection,} is identified where Nu decreases with respect to purely forced convection, associated with a lower turbulence production near the walls. An empirical correlation for the Darcy friction factor ($f$) is also proposed, showing good agreement with data and references.

The laminar–turbulent transition is examined for $\mathrm{Re}_o=5000$, $\mathrm{Pr}=0\text{.}71$, and two buoyancy intensities ($\mathrm{Ri}_b=0\text{.}04$ and $\mathrm{Ri}_b=1\text{.}06$). Combinations of perturbations that induce transition are identified, and it is observed that higher buoyancy increases the flow instability. The TKE \textcolor{black}{(Turbulent Kinetic Energy)}, temperature variance, Nu, and $f$ exhibit non-monotonic transient responses; in the case with higher $\mathrm{Ri}b$, a sharp decrease in Nu is recorded, concurrent with an increase in TKE and a flattening of the velocity profile due to turbulence. Finally, it is observed that the value of the developed turbulent-state $\mathrm{Re}\tau$ \textcolor{black}{(friction Reynolds number)} lies above the value associated with the initial state for the lower $\mathrm{Ri}_b$ case and below for the higher $\mathrm{Ri}_b$ case.


\end{abstract}



%%% Local Variables: 
%%% mode: latex
%%% TeX-master: "template"
%%% End: 
