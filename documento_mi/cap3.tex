\chapter{Resultados de Simulaciones}




\newpage

\section{Validaciones de Simulaciones de Canales Periódicos con Convección Forzada}

\subsection{Distintos valores de Prandt}

\begin{figure}[H]
 \centering
  \subfloat[]{
    \includegraphics[width=0.49\textwidth]{results/kawamura/prandts/tep_theta.png}
    \label{fig:phi_mean_kawa}}  
    \subfloat[]{
    \includegraphics[width=0.49\textwidth]{results/kawamura/prandts/tep_theta_log.png}
    \label{fig:phi_mean_log_kawa}}  
 \caption{Perfiles de temperatura media $\langle  \theta^+ \rangle$ en unidades de pared. a) Escala lineal. b) Escala logaritmica en en y.} 
 \label{fig:kawamura_1}
\end{figure}

\begin{figure}[H]
 \centering
  \subfloat[]{
    \includegraphics[width=0.33\textwidth]{results/kawamura/prandts/tep_thetap_rms.png}
    \label{fig:phi_rms_kawa}}  
  \subfloat[]{
    \includegraphics[width=0.33\textwidth]{results/kawamura/prandts/tep_up_thetap.png}
    \label{fig:phi_up_thetap_kawa}}
  \subfloat[]{
    \includegraphics[width=0.33\textwidth]{results/kawamura/prandts/tep_vp_thetap.png}
    \label{fig:phi_vp_thetap_kawa}}  
    
   \caption{a) Fluctuaciones de la temperatura.  b) Flujo turbulento de calor en la dirección X. c) Flujo turbulento de calor en la dirección Y.} 
 \label{fig:kawamura_2}
\end{figure}

\subsection{Comparación de mallas para el caso $\mathbf{Pr=0.71}$}

\begin{figure}[H]
 \centering
  \subfloat[]{
    \includegraphics[width=0.49\textwidth]{results/kawamura/meshes/tep_theta.png}
    \label{fig:phi_mean_kawa}}  
    \subfloat[]{
    \includegraphics[width=0.49\textwidth]{results/kawamura/meshes/tep_thetap_rms.png}
    \label{fig:phi_rms_kawa}}  
 \caption{a) Perfiles de temperatura media $\langle  \theta^+ \rangle$ en unidades de pared. b) Fluctuaciones de la temperatura.} 
 \label{fig:kawamura_3}
\end{figure}


\begin{figure}[H]
 \centering
  \subfloat[]{
    \includegraphics[width=0.49\textwidth]{results/kawamura/meshes/tep_up_thetap.png}
    \label{fig:phi_up_thetap_kawa}}
  \subfloat[]{
    \includegraphics[width=0.49\textwidth]{results/kawamura/meshes/tep_vp_thetap.png}
    \label{fig:phi_vp_thetap_kawa}}  
    
   \caption{a) Flujo turbulento de calor en la dirección X. b) Flujo turbulento de calor en la dirección Y.} 
 \label{fig:kawamura_4}
\end{figure}


\section{Validación de Simulaciones de Canales Periódicos con Convección Mixta}

\subsection{Caso $\mathbf{Ri_b}=0.5$}

\begin{figure}[H]
 \centering

  \subfloat[]{
    \includegraphics[width=0.49\textwidth]{results/guo/rib05/mct_upmean.png}
    \label{fig:phi_mean_guo}}  
  \subfloat[]{
    \includegraphics[width=0.49\textwidth]{results/guo/rib05/mct_uprms.png}
    \label{fig:phi_rms_guo}} 
 
  \subfloat[]{
    \includegraphics[width=0.49\textwidth]{results/guo/rib05/mct_theta.png}
    \label{fig:phi_mean_guo}}  
  \subfloat[]{
    \includegraphics[width=0.49\textwidth]{results/guo/rib05/mct_thetap_rms.png}
    \label{fig:phi_rms_guo}}  

  \subfloat[]{
    \includegraphics[width=0.49\textwidth]{results/guo/rib05/mct_up_thetap.png}
    \label{fig:phi_up_thetap_guo}}
  \subfloat[]{
    \includegraphics[width=0.49\textwidth]{results/guo/rib05/mct_vp_thetap.png}
    \label{fig:phi_vp_thetap_guo}}  
    
   \caption{a) Perfiles de temperatura media. b) Fluctuaciones de la temperatura. c) Perfiles de la velocidad media en la dirección de la corriente. d) Fluctuaciones de la velocidad en la dirección de la corriente. e) Flujo turbulento de calor en la dirección X. f) Flujo turbulento de calor en la dirección Y.} 
 
 \label{fig:guo}
\end{figure}


\subsubsection{Variación de dominio}

\begin{figure}[H]
 \centering

  \subfloat[]{
    \includegraphics[width=0.49\textwidth]{results/guo/rib05_domains/mct_upmean.png}
    \label{fig:phi_mean_guo}}  
  \subfloat[]{
    \includegraphics[width=0.49\textwidth]{results/guo/rib05_domains/mct_uprms.png}
    \label{fig:phi_rms_guo}} 
 
  \subfloat[]{
    \includegraphics[width=0.49\textwidth]{results/guo/rib05_domains/mct_theta.png}
    \label{fig:phi_mean_guo}}  
  \subfloat[]{
    \includegraphics[width=0.49\textwidth]{results/guo/rib05_domains/mct_thetap_rms.png}
    \label{fig:phi_rms_guo}}  

  \subfloat[]{
    \includegraphics[width=0.49\textwidth]{results/guo/rib05_domains/mct_up_thetap.png}
    \label{fig:phi_up_thetap_guo}}
  \subfloat[]{
    \includegraphics[width=0.49\textwidth]{results/guo/rib05_domains/mct_vp_thetap.png}
    \label{fig:phi_vp_thetap_guo}}  
    
   \caption{a) Perfiles de temperatura media. b) Fluctuaciones de la temperatura. c) Perfiles de la velocidad media en la dirección de la corriente. d) Fluctuaciones de la velocidad en la dirección de la corriente. e) Flujo turbulento de calor en la dirección X. f) Flujo turbulento de calor en la dirección Y.} 
 
 \label{fig:guo}
\end{figure}

\subsection{Caso $\mathbf{Ri_b}=0.25$}

\begin{figure}[H]
 \centering

  \subfloat[]{
    \includegraphics[width=0.49\textwidth]{results/guo/rib025/mct_upmean.png}
    \label{fig:phi_mean_guo}}  
  \subfloat[]{
    \includegraphics[width=0.49\textwidth]{results/guo/rib025/mct_uprms.png}
    \label{fig:phi_rms_guo}} 
 
  \subfloat[]{
    \includegraphics[width=0.49\textwidth]{results/guo/rib025/mct_theta.png}
    \label{fig:phi_mean_guo}}  
  \subfloat[]{
    \includegraphics[width=0.49\textwidth]{results/guo/rib025/mct_thetap_rms.png}
    \label{fig:phi_rms_guo}}  

  \subfloat[]{
    \includegraphics[width=0.49\textwidth]{results/guo/rib025/mct_up_thetap.png}
    \label{fig:phi_up_thetap_guo}}
  \subfloat[]{
    \includegraphics[width=0.49\textwidth]{results/guo/rib025/mct_vp_thetap.png}
    \label{fig:phi_vp_thetap_guo}}  
    
   \caption{a) Perfiles de temperatura media. b) Fluctuaciones de la temperatura. c) Perfiles de la velocidad media en la dirección de la corriente. d) Fluctuaciones de la velocidad en la dirección de la corriente. e) Flujo turbulento de calor en la dirección X. f) Flujo turbulento de calor en la dirección Y.} 
 
 \label{fig:guo}
\end{figure}
