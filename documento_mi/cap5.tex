\chapter{Transición Temporal}




\newpage

\section{Analisis de Estabilidad Lineal}

La transición laminar-turbulenta, es decir, la evolución de un flujo laminar a uno turbulento, es crucial en ingeniería, ya que las características del flujo varían notablemente entre estos regímenes. Por ejemplo, los coeficientes de fricción y de convección aumentan considerablemente al pasar de un régimen laminar a uno turbulento. La ecuación de Navier–Stokes admite ambas soluciones bajo ciertos parámetros, lo que implica que el tipo de flujo y su evolución dependen de las perturbaciones y las condiciones impuestas en el sistema. Muchos fenómenos que cumplen exactamente las leyes de conservación resultan inobservables porque se inestabilizan ante las pequeñas perturbaciones inevitables en cualquier sistema real \cite{kundu}.

El análisis de estabilidad lineal permite evaluar cómo se comporta un flujo ante perturbaciones, identificando los mecanismos que pueden inducir transiciones o estados de intermitencia. En el caso de flujos de fluidos, condiciones como un número de Reynolds inferior a un valor crítico garantizan la estabilidad de un flujo laminar suave. Sin embargo, en ocasiones las perturbaciones crecen hasta alcanzar amplitudes finitas y establecer nuevos equilibrios estacionarios, que pueden volverse inestables a su vez y evolucionar hacia estados de fluctuaciones caóticas, comúnmente descritos como turbulencia. Dos motivaciones principales para estudiar la estabilidad de los fluidos son comprender el proceso de transición de un flujo laminar a uno turbulento y predecir el inicio de dicha transición.

El enfoque parte de las ecuaciones de gobierno \ref{eq:gob_system_adim}. La idea consiste en suponer que los campos solución ($\mathbf{u^*}$,$\text{p}^*$,$\theta^*$) pueden descomponerse como un flujo base más una perturbación:

\begin{align}
\mathbf{u^*} &= \mathbf{U} + \tilde{\mathbf{u}}^* \\
\text{p}^* &= \text{P}+ \tilde{\text{p}}^* \\
\theta^* &= \Theta + \tilde{\theta}^*
\end{align}  
donde las letras mayusculas hacen referencia al flujo base y aquellas letras con $\tilde{()}$ a las perturbaciones. Asimismo, se asume que $\mathbf{u^*}$, $\text{p}^*$, $\mathbf{U} = (U_x,U_y,U_z)$ y $\text{P}$ satisface el sistema \ref{eq:gob_system_adim}. Así, despreciando

\\
