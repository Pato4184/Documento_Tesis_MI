\chapter{Conclusiones} \label{cap:conclusiones}

En este trabajo se presentó el problema de un flujo turbulento asistido por fuerzas \linebreak boyantes en un canal vertical de placas paralelas sometido a un flujo de calor constante en las paredes. Este tipo de sistemas aparece en numerosos dispositivos termohidráulicos, por lo que el estudio de la convección mixta reviste gran importancia en ingeniería. En este contexto, se expuso la formulación matemática que rige los principios de conservación de masa, momento y energía, junto con las condiciones de borde empleadas para el sistema bajo estudio. Dado que se analiza la transición temporal desde un régimen laminar hacia uno turbulento, se introducen las ecuaciones de Orr-Sommerfeld derivadas de la teoría de \linebreak estabilidad lineal \cite{chen1996linear}. Estas constituyen el puntapié inicial de la metodología utilizada para construir perturbaciones capaces de inducir la transición. La resolución del problema de autovalores y autofunciones se realizó con la herramienta OSMC desarrollada en el grupo MECOM \cite{szuban2023}. El análisis de flujos completamente desarrollados y de la transición temporal se efectuó mediante simulación numérica directa (DNS) con Xcompact3D \cite{bartholomew2020xcompact3d}. Ambas herramientas fueron validadas previamente con casos de referencia disponibles en la literatura.

Para estudiar la transición temporal resulta necesario conocer el estado inicial laminar y el estado final turbulento. En consecuencia, se analizó el flujo turbulento completamente desarrollado bajo la influencia de la fuerza boyante mediante simulaciones DNS, considerando distintos valores de los números adimensionales de Reynolds, Prandtl y Richardson. Para $\text{Re}_o=5000$ y $\text{Pr}=0\text{.}71$ se evaluaron magnitudes estadísticas de primer y segundo orden. Primero se observó que para $\text{Ri}_b \geq 1\text{.}06$ los perfiles de velocidad adoptan una forma \linebreak característica en ``M'', en consistencia con otros trabajos \cite{you2003direct}, \cite{zhou2024direct}. Segundo, los perfiles de temperatura se distinguen en dos grupos según $\text{Ri}_b$: para $0 \lesssim \text{Ri}_b \lesssim 1$ se ubican por encima del caso puramente forzado, mientras que para $\text{Ri}_b \gtrsim 1$ quedan por debajo, debido a la mezcla inducida por la flotación. Tercero, manteniendo $\text{Re}_o=5000$, se analizó el efecto de $\text{Pr}$: para $\text{Pr}=0\text{.}071$ la ley de pared \cite{kawamura1998dns} es válida hasta $y^+ \approx 30$, y para $\text{Pr}=0\text{.}71$ hasta $y^+ \approx 7$, evidenciando la influencia de $\text{Pr}$ en la capa conductiva. Cuarto, a partir del conjunto de simulaciones se calculó el número de Nusselt en función del número de boyancia Bo y se lo comparó con la correlación de Jackson et al. \cite{jackson1989studies}, encontrándose un muy buen acuerdo. Se identificó además la existencia de un intervalo $10^{-6} \lesssim \text{Bo} \lesssim 3 \times 10^{-5}$ donde Nu se reduce respecto del caso puramente forzado, señalando una caída en la transferencia de calor que coincide con la disminución de la producción total de turbulencia, principalmente cerca de las paredes. Quinto, se calculó el factor de fricción de Darcy y, para el rango de parámetros considerado, se propuso una correlación tipo potencia que mostró buen acuerdo tanto con nuestras simulaciones como con datos reportados en \cite{you2003direct} y \cite{parlatan1996buoyancy}. Pese a la asistencia de boyancia, el gradiente de velocidad aumenta en la vecindad de la pared y eleva el factor de fricción.

Con el entendimiento del flujo base y del mecanismo de inestabilización utilizado para transicionar el sistema, y con el estado final turbulento caracterizado, se encaró el estudio de la transición temporal laminar-turbulenta. Dado que en la bibliografía reciente este fenómeno cuenta con escasa información, se realizó primero una exploración numérica para identificar combinaciones de perturbaciones capaces de inducir la inestabilidad. Se consideró $\text{Re}_o=5000$ y $\text{Pr}=0\text{.}71$, seleccionando dos valores moderados de $\text{Ri}_b$: $0\text{.}04$ para los Casos A y $1\text{.}06$ para los Casos B. Empleando el mecanismo de inestabilización descrito en el Capítulo \ref{cap:modelo} se construyeron distintas condiciones iniciales. A partir de la evolución temporal de la energía cinética turbulenta y del número de Reynolds de fricción se identificaron las condiciones que efectivamente transicionan el flujo y se eligieron dos ensayos para un análisis detallado: A-C10, cuya condición inicial combina ondas 2D/3D ($c_{\text{2D}}=0\text{.}385 - 0\text{.}124 j$ y $c_{\text{3D}}=0\text{.}563 - 0\text{.}095 j$), y B-C2, con una única onda 2D ($c_{\text{2D}}=0\text{.}800 - 0\text{.}495 j$). Este último resultado indica que la fuerza boyante incrementa la inestabilidad del sistema, de modo que ciertos tipos de perturbaciones inducen antes la transición. A lo largo de la evolución, los perfiles de velocidad y temperatura exhiben pérdidas momentáneas de simetría asociadas a la distribución no homogénea de vórtices cerca de las paredes. El análisis temporal de TKE, varianza de temperatura, número de Nusselt y coeficiente de fricción de Darcy muestra estados transitorios no monótonos, con máximos y mínimos que superan los valores de la condición inicial y del estado turbulento desarrollado. En particular, el ensayo B-C2 presenta una caída brusca de Nu que coincide con el aumento de TKE y el aplanamiento del perfil de velocidad en el mismo instante, fenómeno atribuible a la mayor difusión de momento por efecto de la turbulencia. Si bien el desarrollo térmico se \linebreak acelera frente al aumento de la boyancia, en el tiempo de simulación considerado Nu no alcanzó el estado completamente desarrollado en ninguno de los dos casos analizados en profundidad. Por último, al comparar la evolución de $\text{Re}_{\tau}$ se observa que en el ensayo A-C10 el valor final queda por encima del inicial, mientras que en B-C2 desciende por debajo. Esta diferencia se explica por la variación del gradiente de velocidad en la región próxima a la pared, vinculada al aplanamiento de los perfiles y al incremento súbito de turbulencia.

En síntesis, se llevaron a cabo simulaciones DNS de flujos en régimen de convección mixta con foco en la transición temporal en canales verticales y en su impacto sobre la \linebreak transferencia de calor bajo la acción de la boyancia. Los datos de los casos completamente \linebreak desarrollados pueden emplearse para el diseño de dispositivos termohidráulicos que eviten operar en la región de reducción de transferencia de calor o, en su defecto, para considerar dicho \linebreak comportamiento en su operación. Por su parte, los resultados de transición temporal constituyen un puntapié inicial para el estudio de este fenómeno complejo, aporta al entendimiento del proceso de transición y ofrece una base para contrastar y verificar correlaciones existentes o propuestas en un trabajo futuro.
