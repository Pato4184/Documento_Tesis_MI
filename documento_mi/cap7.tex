\chapter{Conclusiones} \label{cap:conclusiones}

En este trabajo se presentó el problema de un flujo turbulento asistido por fuerzas \linebreak boyantes en un canal vertical de placas paralelas sometido a un flujo de calor constante en las paredes. Este tipo de sistemas aparece en numerosos dispositivos termohidráulicos, por lo que el estudio de la convección mixta reviste gran importancia en ingeniería. Entre aquellos sistemas de interés se hallan los intercambiadores de calor, los cuales se encuentran integrados a su vez en sistemas más complejos, como por ejemplo centrales termoeléctricas. El objetivo de este trabajo consiste en estudiar la transferencia de calor en régimen de transición laminar-turbulenta en convección mixta. Se calculan y evalúan magnitudes de interés como por ejemplo el número de Nusselt y/o el factor de fricción de Darcy.

En este contexto, se expuso la formulación matemática que rige los principios de conservación de masa, momento y energía, junto con las condiciones de borde empleadas para el sistema bajo estudio. Dado que se analiza la transición temporal desde un régimen \linebreak laminar hacia uno turbulento, se introducen las ecuaciones de Orr-Sommerfeld derivadas de la teoría de estabilidad lineal \cite{chen1996linear}. Su resolución conduce a un problema de autovalores y autofunciones. Estas constituyen el punto de partida de la metodología utilizada para construir perturbaciones capaces de inducir la transición. El problema de autovalores y autofunciones se transformó en un problema de autovalores y autovectores empleando la herramienta OSMC desarrollada en el grupo MECOM \cite{szuban2023}. El análisis de flujos completamente desarrollados y de la transición temporal se efectuó mediante simulación numérica directa (DNS) con Xcompact3D \cite{bartholomew2020xcompact3d}. Ambas herramientas fueron validadas previamente con casos de referencia disponibles en la literatura.

Para estudiar la transición temporal resulta necesario conocer el estado inicial laminar y el estado final turbulento. En consecuencia, se analizó el flujo turbulento completamente desarrollado bajo la influencia de la fuerza boyante mediante simulaciones DNS, considerando distintos valores de los números adimensionales de Reynolds, Prandtl y Richardson. Para $\text{Re}_o=5000$ y $\text{Pr}=0\text{.}71$ se evaluaron magnitudes estadísticas de primer y segundo orden. A continuación, se resumen las observaciones principales:

\begin{itemize}

\item Para $\text{Ri}_b \geq 1\text{.}06$ los perfiles de velocidad adoptan una forma característica en ``M'', en consistencia con otros trabajos \cite{you2003direct}, \cite{zhou2024direct}.

\item Los perfiles de temperatura se distinguen en dos grupos según $\text{Ri}_b$: para $0 \lesssim \text{Ri}_b \lesssim 1$ se ubican por encima del caso puramente forzado, mientras que para $\text{Ri}_b \gtrsim 1$ quedan por debajo, debido a la mezcla inducida por la flotación.

\item Manteniendo $\text{Re}_o=5000$, se analizó el efecto de $\text{Pr}$. Para $\text{Pr}=0\text{.}071$, la ley de pared \cite{kawamura1998dns} es válida hasta $y^+ \approx 30$; para $\text{Pr}=0\text{.}71$, hasta $y^+ \approx 7$, evidenciando la influencia de $\text{Pr}$ en la capa conductiva.

\item A partir del conjunto de simulaciones se calculó el número de Nusselt en función del número de boyancia Bo y se lo comparó con la correlación de Jackson et al. \cite{jackson1989studies}, encontrándose un muy buen acuerdo. Se identificó además la existencia de un intervalo $10^{-6} \lesssim \text{Bo} \lesssim 3 \times 10^{-5}$ donde Nu se reduce respecto del caso puramente forzado, señalando una caída en la transferencia de calor que coincide con la disminución de la producción total de turbulencia, principalmente cerca de las paredes.

\item Se calculó el factor de fricción de Darcy y, para el rango de parámetros considerado, se propuso una correlación tipo potencia que mostró buen acuerdo tanto con nuestras simulaciones como con datos reportados en \cite{you2003direct} y \cite{parlatan1996buoyancy}. Aún con la asistencia de la boyancia, el gradiente de velocidad en la vecindad de la pared aumenta, lo que incrementa el factor de fricción

\end{itemize} 

Tomando las soluciones laminares del flujo bajo análisis (véase Sección \ref{sec:fbase}) \cite{tao1960, chen1996linear}, considerando el mecanismo de inestabilización obtenido mediante teoría de estabilidad lineal \cite{schlatter2005} y con el estado final turbulento ya caracterizado, se estudió la transición temporal laminar–turbulenta. Dado que en la bibliografía reciente este fenómeno cuenta con escasa información, se realizó primero una exploración numérica para identificar combinaciones de perturbaciones capaces de inducir la inestabilidad. Se consideró $\text{Re}_o=5000$ y $\text{Pr}=0\text{.}71$, seleccionando dos valores moderados de $\text{Ri}_b$: $0\text{.}04$ para los Casos A y $1\text{.}06$ para los Casos B. Empleando el mecanismo de inestabilización descrito en el Capítulo \ref{cap:modelo} se construyeron distintas condiciones iniciales. A partir de la evolución temporal de la energía cinética turbulenta y del número de Reynolds de fricción se identificaron las condiciones que efectivamente transicionan el flujo y se eligieron dos ensayos para un análisis detallado: A-C10, cuya condición inicial combina ondas 2D/3D ($c_{\text{2D}}=0\text{.}385 - 0\text{.}124 j$ y $c_{\text{3D}}=0\text{.}563 - 0\text{.}095 j$), y B-C2, con una única onda 2D ($c_{\text{2D}}=0\text{.}800 - 0\text{.}495 j$). Este último resultado indica que la fuerza boyante incrementa la inestabilidad del sistema. 

En este punto, es importante destacar que se implementó satisfactoriamente una \linebreak metodología que permite inducir la transición temporal del flujo. Esto da lugar a potenciales estudios detallados de esta evolución temporal en trabajos futuros. Respecto a lo obtenido en los dos escenarios de transición se puede resumir lo siguiente:

\begin{itemize}

\item La evolución temporal de la TKE, la varianza de temperatura, el número de Nusselt y el coeficiente de fricción de Darcy presentan estados transitorios no monótonos, con \linebreak máximos y mínimos que superan los valores de la condición inicial y del estado turbulento desarrollado.

\item En el ensayo A-C10, Nu permanece cercano al valor del estado laminar y luego crece monótonamente. En cambio, en el ensayo B-C2, se presenta una caída brusca que \linebreak coincide con el aumento de la TKE y el aplanamiento del perfil de velocidad en el mismo instante, fenómeno atribuible a la mayor difusión de momento por efecto de la \linebreak turbulencia. 

\item Al considerar ambas condiciones iniciales, el valor de Nu tiende al estado turbulento desarrollado pero no lo alcanza (al menos en la ventana de tiempo simulada). 

\item Al comparar la evolución de $\text{Re}_{\tau}$ se observa que en el ensayo A-C10 el valor de $\text{Re}_{\tau}$  en el estado turbulento desarrollado queda por encima del inicial, mientras que en B-C2 ocurre lo contrario. Esta diferencia se explica por la variación del gradiente de velocidad en la región próxima a la pared, vinculada al aplanamiento de los perfiles y al incremento súbito de turbulencia.

\item A lo largo de la evolución temporal de ambos ensayos, los perfiles de velocidad y temperatura exhiben pérdidas momentáneas de simetría asociadas a la distribución no homogénea de vórtices cerca de las paredes. Este efecto podría ser causado, posiblemente, por el tamaño del dominio empleado.

\end{itemize}

En síntesis, se llevaron a cabo simulaciones DNS de flujos en régimen de convección mixta con foco en la transición temporal en canales verticales y en su impacto sobre la \linebreak transferencia de calor bajo la acción de la boyancia. Los datos de los casos completamente \linebreak desarrollados pueden emplearse para el diseño de dispositivos termohidráulicos que eviten \linebreak operar en la región de reducción de transferencia de calor o, en su defecto, para considerar \linebreak dicho  comportamiento en su operación. Por su parte, los resultados de transición temporal constituyen un punto de partida para el estudio de este fenómeno complejo, aportan al entendimiento del proceso de transición y ofrece una base para contrastar y verificar correlaciones existentes o propuestas en un trabajo futuro.
